%% Chapter without number, but included in header and TOC
\NewDocumentCommand{\chapternotnumbered}{m}{
    \chapter*{#1}
    \markboth{#1}{#1}                   % use chapter title in header
    \addcontentsline{toc}{chapter}{#1}  % include chapter in TOC
}

%%% Chapter and section formatting
\usepackage{titlesec}

\titleformat{\section}{\normalfont\Large\bfseries}{\thesection}{0.8em}{}
%% NOTE: also include `\phantomsection` so even non-numbered subsections have proper anchors for hyperlinks
\titleformat{\subsection}{\phantomsection\normalfont\large\bfseries}{\thesubsection}{0.8em}{}

%%% Chapter title formatting
%% No extra space above and below chapter headings, big number/letter behind chapter title
%% inspired by https://tex.stackexchange.com/a/690632
%% TODO: try to shift chapter anchors a little bit up, so reference links give a nice view
\NewDocumentCommand{\chaphead}{m O{}}{
    {\setlength{\parindent}{0pt} \raggedright
        % \vspace*{15pt}%
        \Huge\bfseries%
        \ifFANCY%
            \chapterheadingletter{#2}% fancy Chapter number/letter
        \else%
            \IfBlankF{#2}{\thechapter\hspace{0.7em}}% just basic Chapter number
        \fi%
        #1% Chapter title
        \par\nobreak
        \vspace{20pt}
    }
}
\makeatletter
\def\@makechapterhead#1{\chaphead{#1}[\thechapter]} % For numbered Chapters use their number
\def\@makeschapterhead#1{
    \ifFANCY
        \chaphead{#1}[\extract{#1}{1}] % extract first letter of the current chapter title
    \else
        \chaphead{#1} % just Chapter title
    \fi
}
\makeatother

%% Chapter number/letter formatting
\NewDocumentCommand{\chapterheadingletter}{m}{%
    \makebox[0pt][l]{%              Make box of zero width, don't move other stuff horizontally
        \raisebox{-16pt}[0pt][0pt]{% Align the number vertically, don't move other stuff vertically
            \hspace{6pt} \color{ChapterNumberColor}% Horizontal whitespace, text color
            \usefont{T1}{qzc}{m}{it}\fontsize{95pt}{95pt}\selectfont% Font type and size (TeX Gyre Chorus)
            #1 % Chapter heading letter
        }%
    }%
}
%% Macro to extract first `#2` characters from `#1`
%% https://tex.stackexchange.com/questions/402835/extract-first-character-of-string-stored-in-macro-using-expl3
%% TODO: latex treesitter grammar support ExplSyntaxOn/Off, command names containting also `_:`
\ExplSyntaxOn
\cs_generate_variant:Nn \tl_item:nn { f }
\DeclareExpandableDocumentCommand{\extract}{mm}{
    \tl_item:fn { #1 } { #2 }
}
\ExplSyntaxOff
