% \setcounter{chapter}{16} % "Q" (17th letter, but latex first increments -> 17-1=16)
% \chapter{Résumé Rapide}
\chapternotnumbered{Résumé Rapide} \label{ch:Quick Summary}

\vspace{2ex} % espace vertical supplémentaire, car la lettre Q a une longue queue

\textbf{Comparaison et ordre de grandeur \Cref{comp/ord}} : Traite des comparaisons, des ordres de grandeur.

\textbf{Developpement et series de Taylor \Cref{taylor}}

\textbf{Intégrales Généralisées \Cref{int}} : Couvre les concepts d’intégrales étendus au-delà des définitions standard et leur importance dans la résolution de problèmes.

\textbf{Séries Numériques \Cref{nseries}} : Explore les propriétés des séries numériques, y compris les tests de convergence et des exemples.

\textbf{Séries Entières \Cref{pseries}} : Examine le développement et l’analyse des séries entières, en mettant l’accent sur leur convergence et leur utilité dans les applications.

\textbf{Équations Différentielles \Cref{df}} : Explique les méthodes de résolution des équations différentielles du premier et du second ordre avec des coefficients constants, en se concentrant sur les applications pratiques.

\textbf{Systèmes de Coordonnées \Cref{coord}} : Détaille les transformations entre les systèmes de coordonnées cartésiennes, cylindriques et polaires, en mettant en avant leur pertinence dans divers contextes.

\textbf{Séries de Fourier \Cref{fourier}} : Décrit le processus de calcul des coefficients de Fourier, les propriétés de convergence et des exemples pratiques de développements en séries.
