\chapter{Features of \TeXtured{}} \label{ch:Features}

\begin{Note}[WIP Disclaimer]
    This chapter is Work-in-Progress.
\end{Note}

First two sections are about typography of the document, next are more about its contents and structure.

\section{Page Layout and Style}%
\label{sec:Page Layout}

\subsection{Page Dimensions, Printing Layout}%
\label{sub:Page Dimensions}

\package{geometry} package --- set up the page layout (supported single/double-sided printing).
Apply \macro{\flushbottom} --- try to make text body on all pages have the same height.

\subsection{Headers, Footers}%
\label{sub:Headers Footers}

\package{fancyhdr} package --- headers and footers --- consistent style also for initial page of a chapter (not totally different style with numbering in the bottom center \ldots).

\subsection{Heading Style}%
\label{sub:Heading Style}

Pretty chapter heading style --- big calligraphic number/letter behind the title.


\section{Sane Typographical Defaults}%
\label{sec:Sane Typographical Defaults}

Now we will concern ourselves with more intricate and detailed typography (more at level of sentences, words, and even letters).

\subsection{Paragraphs, Captions}%
\label{sub:Paragraphs}

No paragraph indentation, proper space between paragraphs --- \package{parskip}.
Caption styling --- hang, \macro{\footnotesize} font, bold serif label (like other environments).

\subsection{Font and Related Stuff}%
\label{sub:Font}

The default choice are \emph{Latin Modern} fonts --- a classic really.
\begin{itemize}
    \item \emph{Serif} family for the main text
    \item \emph{Slanted} shape for emphasis (instead of the default \emph{Italic} shape, which is reserved mainly for math formulas)
    \item \emph{(Bold) Sans} family for headings and other structural elements
    \item \emph{Typewriter} family for computer code and similar stuff
\end{itemize}
For consistent quotation use \macro{\enquote} macro provided by \package{csquotes}.

\subsection{\texorpdfstring{Micro-\!Typography}{Micro-Typography}}%
\label{sub:Micro-Typography}

Enable micro-typographic extensions with package \package{microtype}, most prominently character protrusion and font expansion.

Following quote from \package{microtype} documentation nicely explains what it is about:
\begin{displayquote}
    Micro-typography is the art of enhancing the appearance and readability of a
    document while exhibiting a minimum degree of visual obtrusion.
    It is concerned with what happens between or at the margins of characters, words or lines.
    Whereas the macro-typographical aspects of a document (i.e., its layout) are clearly visible even to the untrained eye, micro-typographical refinements should ideally not even be recognizable.
    That is, you may think that a document looks beautiful, but you might not be able to tell exactly why: good micro-typographic practice tries to reduce all potential irritations that might disturb a reader.
\end{displayquote}


\section{Document Structure}%
\label{sec:Document Structure}

Implemented mostly via \package{tcolorbox} package (and \package{thmtools}).

\subsection{Theorem/Remark--like Environments}%
\label{sub:Theorem/Remark--like Environments}

\begin{itemize}
    \item clear structure, enables high level of interlinking
    \item include title if possible --- rough idea after first look, quicker navigation/finding
    \item non-intrusive indicators for the end of non-boxed environments
    \item shared numbering, together with tables, figures, equations, \ldots
          --- linear increase of reference number, easier to find
    \item easy to skim through, get quickly an idea, and know what to expect
    \item not only for physics/math texts, can be used to highlight key ideas
    \item no emphasis for the whole body of definitions \ldots we have box for that \ldots
\end{itemize}

\begin{remark}
    There are also helper environments for \textsf{Todo}-like notes.
    By default, there are \textsf{Todo}, \textsf{Note}, \textsf{Suggestion}, and \textsf{Question} environments, but you can easily create your own.
\end{remark}

\begin{Note}
    No \enquote{code listing} setup yet.
    PRs welcome.
\end{Note}

\subsection{References, Links}%
\label{sub:References Links}

Custom reference/link/citation styles.
\begin{itemize}
    \item I like distinguished references/links/citations
    \item always use \macro{\Cref} from \package{cleveref} package
    \item frames are useful, but not that pretty (they can obscure punctuation \ldots),
          and the PDF viewer must support them
    \item just color is sometimes alright, but less ideal when printing
    \item sometimes good to have an idea where they point (mainly when printed, no ability to click) --- include page number
    \item there is slight inconvenience due to small flexibility around line breaks (would be nice to have proper workaround, like link breaking or something like that)
\end{itemize}

\subsection{Table of Contents, Index}%
\label{sub:Table of Contents and Index}

Clear Table of Contents.
\begin{itemize}
    \item subsections included (in a compact style), but not numbered...
    \item don't frame references in the Contents, just keep it clean and simple
    \item include everything important
\end{itemize}

Proper Index (ToC in PDF viewer).
\begin{itemize}
    \item very handy for navigating longer documents
    \item include also other important pages other than just initial pages of main chapters: Contents, Introduction, References, \ldots
    \item I use \texttt{Zathura} as my PDF viewer, where Index is just one \macro{Tab} away \ldots
\end{itemize}


\section{Bibliography/References}%
\label{sec:Bibliography/References}

Clear Bibliography/References, via \package{biblatex} package.

\subsection{Bibliography Style}%
\label{sub:Bibliography Style}

Entries in \Nref{ch:References}:
\begin{itemize}
    \item have a clean consistent style based on the \macro{ext-numeric-verb} style from \package{biblatex-ext} package
    \item support (new) \custommacro{github} field
    \item have DOI/arXiv/URL/GitHub links on the new line
    \item include \enquote{backlinks}
\end{itemize}

\begin{remark}[Bibliography Data]
    Make sure to gather all the relevant data you need for every reference.
    If you later decide you want to reduce the amount of presented information, \package{biblatex} can help you with that.
    For example, it is possible to automatically
    \begin{itemize}
        \item remove \macro{url} field if \macro{doi} field is present,
        \item ignore unwanted fields (\macro{pages}, \macro{number}, \macro{volume}, \macro{series}, \macro{location}, \ldots). \qedhere*
    \end{itemize}
\end{remark}

\subsection{Citation Style}%
\label{sub:Citation Style}

Include \texttt{[} and \texttt{]} characters around citation number inside the link (and wrap in \package{tcolorbox} \ldots), for example \autocite{TeXtured}.


\section{PDF/A Compliance}%
\label{sec:PDF/A Compliance}

Proper metadata setup (via \package{hyperref} and \macro{\DocumentMetadata}).
\begin{remark}[Document Data]
    Various data about the work should be entered in \file{preamble/data.tex} file.
    When the relevant entries contain \LaTeX{} commands (for example to obtain specific formatting of the title), it is necessary to provide \enquote{plaintext} variations, so that \package{hyperref} can properly set up PDF metadata.
\end{remark}

Next we will describe various common violations of PDF/A standard, and how to fix them.

\subsection{Glyph to Unicode Map}%
\label{sub:Glyph to Unicode Map}

\begin{Todo}
    Link to guide on how to find the name of a glyph, and how to provide \emph{glyph to Unicode} mapping.
\end{Todo}

\subsection{PDF \texorpdfstring{\fakemacro{/Interpolation}}{/Interpolation} Key}%
\label{sub:PDF Interpolation Key}

Some PDFs can have enabled the \macro{/Interpolation} key, for example \texttt{Inkscape} generated PDFs with blur parts.
However, PDF/A requires it to be disabled.


\section{Code Organization}%
\label{sec:Code Organization}

\begin{itemize}
    \item code broken into multiple files by its function, see \Cref{fig:preamble-file-structure}
    \item use \macro{aux_dir} to minimize clutter
    \item avoid large, hard to navigate, preamble files
    \item try to follow \enquote{one sentence per line} rule --- improves diffs for versioning systems like \texttt{git} --- tools like \texttt{latexindent} can help
          \begin{Note}
              My config for \texttt{latexindent} mostly works, but some corner cases can surface.
              Will share someday.
          \end{Note}
\end{itemize}

\begin{remark}[Custom User Macros]
    Enter your own macros in the \path{preamble/user/} directory, which is reserved precisely for this purpose.
    Then, if you would like to update to a newer version of \TeXtured{}, you will be having easier time --- less mixing of your code with the template code will result in fewer conflicts you must resolve manually.
\end{remark}

\begin{figure}[!ht]
    \fcapside[\FBwidth]{%
        \captionsetup{parskip = .5\baselineskip plus 1pt}%
        \caption[File structure of preamble folder]{
            File structure of the \path{preamble/} folder.

            Make sure to fill all (meta)data --- like title, author, etc.~--- in the \path{preamble/data.tex} file.
            The bibliography/reference data is stored in the \path{preamble/bibliography.bib} file.

            Apart from the \path{preamble/pdfA-compliance/glyphtounicode.tex} file, which needs to be sourced before calling \fakemacro{\documentclass}, all other files are loaded in the \path{preamble/main.tex} file.

            When possible, add your own tweaks and macros to the \path{preamble/user/} directory reserved for this purpose.
            This way, you can easily update to newer versions of \TeXtured{} (hopefully) without conflicts.
        }%
        \label{fig:preamble-file-structure}%
    }{%
        \begin{tikzpicture}[
                % http://www.texample.net/tikz/examples/filesystem-tree/
                growth parent anchor=west,
                grow via three points={one child at (1.0,-0.78) and
                    two children at (1.0,-0.78) and (1.0,-1.56)},
                edge from parent path={([xshift=1.2em]\tikzparentnode.south west) |- (\tikzchildnode.west)},
                every node/.style={draw=black, thick, anchor=west, fill=black!5, text depth=.25ex},
                directory/.style={fill=black!13},
                font=\ttfamily,
            ]%
            \node[directory] {preamble/}
            child { node {bibliography.bib}}
            child { node {data.tex}}
            child { node {main.tex}}
            child { node[directory] {debug/...}
                % child { node {commands.tex}}
                % child { node {line-numbers.tex}}
            }
            % child [missing] {} child [missing] {}
            child { node[directory] {environments/...}
                % child { node {init.tex}}
                % child { node {remark-like.tex}}
                % child { node {theorem-like.tex}}
                % child { node {todo-like.tex}}
            }
            % child [missing] {} child [missing] {} child [missing] {} child [missing] {}
            child { node[directory] {general/...}
                % child { node {colors.tex}}
                % child { node {floats.tex}}
                % child { node {hyperref.tex}}
                % child { node {typesetting.tex}}
            }
            % child [missing] {} child [missing] {} child [missing] {} child [missing] {}
            child { node[directory] {hacks/...}
                % child { node {custom-reference-boxes.tex}}
                % child { node {fix-qed.tex}}
                % child { node {floatrow-parskip.tex}}
            }
            % child [missing] {} child [missing] {}
            child { node[directory] {layout/...}
                % child { node {geometry.tex}}
                % child { node {headers.tex}}
                % child { node {numbering.tex}}
                % child { node {titles.tex}}
                % child { node {toc.tex}}
            }
            % child [missing] {} child [missing] {} child [missing] {} child [missing] {} child [missing] {}
            child { node[directory] {math/...}
                % child { node {fonts.tex}}
                % child { node {macros.tex}}
                % child { node {packages.tex}}
                % child { node {spacing.tex}}
            }
            % child [missing] {} child [missing] {} child [missing] {} child [missing] {}
            child { node[directory] {misc/...}
                % child { node {inkscape.tex}}
                % child { node {macros.tex}}
                % child { node {tikz.tex}}
            }
            % child [missing] {} child [missing] {} child [missing] {}
            child { node[directory] {pdfA-compliance/...}
                % child { node {glyphtounicode.tex}}
                % child { node[directory] {LaTeX-find-glyph-name/}
                %     child { node {LaTeX-find-glyph-name.tex}}
                % }
            }
            % child [missing] {} child [missing] {} child [missing] {}
            child { node[directory] {references/...}
                % child { node {backref.tex}}
                % child { node {biblatex-extra-fields.dbx}}
                % child { node {biblatex.tex}}
                % child { node {cite.tex}}
                % child { node {doi-eprint-url.tex}}
                % child { node {fields.tex}}
                % child { node {style.tex}}
            }
            % child [missing] {} child [missing] {} child [missing] {} child [missing] {} child [missing] {} child [missing] {} child [missing] {}
            child { node[directory] {user/...}
                % child { node {macros.tex}}
                % child { node {math.tex}}
            }
            % child [missing] {} child [missing] {}
            ;
        \end{tikzpicture}%
    }
\end{figure}


\section{Miscellaneous}%
\label{sec:Miscellaneous}

\subsection{Math-Related Tweaks --- \texorpdfstring{\(\E^{\I\pi}\)}{exp(iπ)}}%
\label{sub:Math Macros}

Some of the math-related tweaks:
\begin{itemize}
    \item Use \macro{\boldmath} automatically for \macro{\textbf} text (useful mainly in headings).
    \item Possible to use serif italic font for math via \custommacro{\mathsfit}.
    \item Better extendable arrows with \package{TikZ}.
    \item \emph{(Optional, disabled by default)} Automatically change usage of \macro{\textcolor} to \macro{\mathcolor} in math mode, so that we get proper math spacing, for example
          \[ a \mathcolor{gray}{\times} b \text{ (right spacing)} \qquad\text{versus}\qquad a \textcolor{gray}{\times} b \text{ (wrong spacing)}\eqend \]
          However, it is recommended to explicitly use \macro{\mathcolor} when appropriate, since it leads to easier maintenance of the code (copy-pasting to other projects will work without problems).
\end{itemize}

\vspace{1ex}
Following practice is highly recommended.
\begin{tip}[Define Your Own Math Macros]
    Frequently define macros for notation used more than once. Advantages are for example:
    \begin{itemize}
        \item Code is easier to read/write, since it is more \enquote{semantic}.
        \item To tweak notation, you only need to change it in one place.
        \item Easier to find all occurrences of a certain notion. \qedhere
    \end{itemize}
\end{tip}

\subsection{GitHub Actions}%
\label{sub:GitHub Actions}

Describe implemented GitHub Actions:
\begin{itemize}
    \item Automatic \texttt{latexmk} build of the latest PDF version.
    \item PDF/A verification via veraPDF.
    \item Deploy to \texttt{gh-pages} branch.
          One can furthermore enable (in repo settings) GitHub Pages for \texttt{gh-pages} branch, which will automatically upload latest PDF to \texttt{https://username.github.io/reponame/thesis.pdf}.
          This enables convenient sharing of your (even continuously evolving) work without needing to commit the PDF (resulting in large repository size) or compiling the PDF on the receiving side.
          \begin{remark}[Private Repositories]
              Even for private repositories such link is publicly accessible.
              This is why GitHub Pages setup is not done automatically for you.
              If you want to share the work more \enquote{privately}, there are other solutions, for example GitHub Action which uploads PDF to Google Drive, and sharing via a private link.
              Also look at \Cref{sub:Censoring}.
          \end{remark}
\end{itemize}

\subsection{Censoring}%
\label{sub:Censoring}

Censoring/redaction using \package{censor} package.
Use \macro{\censor} or \macro{\censorbox}.
For example, censor the following sentence: \censor{\TeXtured{} is an amazing template!}.

\subsection{\texorpdfstring{\texttt{Inkscape}}{Inkscape} Integration}%
\label{sub:Inkscape Integration}

\begin{itemize}
    \item automatic export after changing the \texttt{svg} (need to enable \macro{--shell-escape} for \hologo{pdfTeX} or \hologo{LuaTeX}, done via \file{.latexmkrc})
    \item watermark via \texttt{ps} injection
          \begin{remark}[Watermark String]
              Edit the watermark text in the shell script \file{figures/Inkscape/inkscape-export-to-latex} to your liking.
          \end{remark}
          \begin{Todo}
              Make it easier to change the watermark text.
              Or extract it somehow automatically from the PDF document?
          \end{Todo}
    \item automatic fix of \macro{/Interpolation} key problem
\end{itemize}


\section{Non-Features}%
\label{sec:Non-Features}

These features were deemed unnecessary, or even counterproductive, and thus were not implemented/not customized.
This does not mean that it is hard or not compatible to use them with \TeXtured{}.

\subsection{Footnotes}%
\label{sub:Footnotes}

\begin{itemize}
    \item they break the flow of reading, can be distracting
    \item either it is important and you want it there --- no need to use footnotes,
          or it is not so important (maybe just a reminder/remark), but then there
          are in my opinion better ways to handle such situation
          \begin{itemize}
              \item grayed out/smaller text, sidenotes are better alternative, if the page layout enables them
              \item it is not bad to remind reader of something in the main text\ldots
          \end{itemize}
\end{itemize}

\subsection{List of Figures, List of ...}%
\label{sub:List of Figures List of ...}

Don't really see the point, but one is free to include it.

\subsection{Index, Glossary}%
\label{sub:Index Glossary}

\begin{itemize}
    \item since the text is primarily intended for electronic use, finding usage of certain terms is easy
    \item text should be ideally structured in such a way, that finding definitions of important terms is straightforward --- interlinking/referencing in proper places to indicate where the notion to be used was defined/discussed
\end{itemize}
