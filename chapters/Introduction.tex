\chapternotnumbered{Introduction} \label{ch:Introduction}

Les mathématiques constituent le pilier de nombreuses disciplines scientifiques et d’ingénierie, fournissant les outils et les méthodes nécessaires pour résoudre des problèmes complexes et modéliser des phénomènes réels. Ce document est une compilation complète de notes et de concepts pour deux cours fondamentaux suivis au cours du troisième semestre du programme du portail ISTN : \emph{Analyse 2 (AN2)} et \emph{Outils Mathématiques (OM)}. Ces cours sont conçus pour s’appuyer sur les bases mathématiques couvertes en première année, dotant les étudiants de compétences analytiques avancées et de résolution de problèmes essentielles pour leur parcours académique et professionnel.

Le cours \emph{Analyse 2 (AN2)} se concentre sur le développement d’une compréhension approfondie des fonctions, des intégrales et des séries. Il met l’accent sur l’application pratique des méthodes de calcul avancées tout en favorisant une approche rigoureuse et précise du raisonnement mathématique. Les sujets tels que \textbf{le comportement des fonctions \Cref{comp/ord}}, \textbf{les sériés de Taylor \Cref{taylor}}, \textbf{les intégrales généralisées \Cref{int}}, \textbf{les séries numériques \Cref{nseries}}, et \textbf{les séries entières \Cref{pseries}}, sont introduits, posant les bases pour des études ultérieures en probabilités, en analyse quantitative et dans d’autres cours avancés du programme. L’approche structurée et méthodique de la résolution de problèmes mise en avant dans ce cours garantit que les étudiants non seulement maîtrisent les concepts théoriques, mais les appliquent également efficacement.

Le deuxième cours, \emph{Outils Mathématiques (OM)}, fournit les techniques mathématiques nécessaires à une large gamme d’applications en physique, en ingénierie et au-delà. En explorant des sujets tels que \textbf{les équations différentielles \Cref{df}}, \textbf{les systèmes de coordonnées \Cref{coord}}, et \textbf{les séries de Fourier \Cref{fourier}}, ce cours dote les étudiants d’outils polyvalents pour analyser et modéliser des systèmes dynamiques. Il comble le fossé entre les connaissances théoriques et leur application pratique, permettant aux étudiants d’aborder des problèmes complexes avec confiance.
