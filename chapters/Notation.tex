\setcounter{chapter}{13} % "N" (14th letter, but latex first increments -> 14-1=13)
\chapter{Notation \& Conventions} \label{ch:notation}
% NOTE: since we are in the "frontmatter" part, we need to manually set headers
%       we didn't use `\chapternotnumbered`, because we want its letter in TOC,
%       and the numbering of environments to be of the form "N.1"
\markboth{Notation \& Conventions}{Notation \& Conventions}

\vspace{2ex}

\begin{example}[Usage of Mathematical Fonts]
    To make the text more readable and beautiful, we can use different types of mathematical fonts for different types of objects (striving to be at least somewhat consistent):
    \begin{itemize}
        \item \(\bm{Bold}\) often for tensorial object (abstract index).
        \item \(\mathsf{Serif}\) for groups, certain spaces, or some operations/maps.
        \item \(\mathfrak{Fraktur}\) for algebras (and densities).
        \item \(\mathcal{C}a\ell\ell i\textsl{g}raphic\) (available are only capital letters, and \(\ell\), also different \textsl{g}).
        \item \(\mathbb{D}\)ouble-\(\mathbb{S}\)truck for fields like \(\R\), spaces like \(\Sphere^{n}\) and \(\CP^{n}\).
        \item \(\mathtt{Typewriter}\) for code functions, or other special objects. \qedhere
    \end{itemize}
\end{example}

\begin{remark}
    This chapter is numbered (or perhaps more precisely \enquote{lettered}).
    This means that is appears in Table of Contents with its letter \textbf{N}, which also prefixes all numbering of environments in this chapter.

    On the other hand, \Nref{ch:Introduction} and \Nref{ch:Quick Summary} are unnumbered (or \enquote{unlettered}) in this sense.
\end{remark}
