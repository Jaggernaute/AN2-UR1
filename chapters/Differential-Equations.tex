\chapter{Équations Différentielles}\label{df}
\section{Résolution des Équations Différentielles}
\subsection{Équations du Premier Ordre}
\begin{itemize}
    \item Forme générale : \( F(x, y, y') = 0 \).
    \item Méthodes de résolution analytique :
    \begin{itemize}
        \item Normalisation.
        \item Résolution des équations sans second membre.
        \item Cas des coefficients constants.
    \end{itemize}
    \item Exemples pratiques.
\end{itemize}

\subsection{Équations du Second Ordre}
\begin{itemize}
    \item Forme générale : \( F(x, y, y', y'') = 0 \).
    \item Méthodes analytiques pour les équations linéaires à coefficients constants.
    \item Applications :
    \begin{itemize}
        \item Mécanique (système masse-ressort, chute libre).
        \item Circuits électriques (RLC).
    \end{itemize}
\end{itemize}

\section{Applications des Équations Différentielles}
\subsection{Mécanique}
\begin{itemize}
    \item Principe fondamental : \(\Sigma \vec{F} = m\vec{a}\).
    \item Exemples :
    \begin{itemize}
        \item Chute libre.
        \item Mouvement amorti (masse-ressort).
    \end{itemize}
\end{itemize}

\subsection{Circuits Électriques}
\begin{itemize}
    \item Cas du premier ordre : décharge d’un condensateur.
    \item Cas du second ordre : oscillations dans un circuit RLC.
\end{itemize}

\section{Méthodes Générales pour les Équations Complètes}
\begin{itemize}
    \item Théorie des solutions homogènes et particulières.
    \item Principe de superposition.
    \item Méthode de variation de la constante de Lagrange.
    \item Exemples : résolution d’équations différentielles avec second membre.
\end{itemize}

\section{Récapitulatif et Étapes Clés pour la Résolution}
\begin{enumerate}
    \item Mise en forme de l’équation.
    \item Recherche des solutions homogènes (\( y_H \)).
    \item Recherche de solutions particulières (\( y_p \)).
    \item Combinaison des solutions : \( y = y_H + y_p \).
    \item Détermination des constantes à partir des conditions initiales.
\end{enumerate}

\section{Exemples Pratiques et Exercices}
\begin{itemize}
    \item Résolution d’équations différentielles avec ou sans second membre.
    \item Applications concrètes (mécanique, circuits, etc.).
\end{itemize}

\section*{Conclusion}
\begin{itemize}
    \item Synthèse des approches méthodologiques.
    \item Importance des équations différentielles dans les applications scientifiques et techniques.
\end{itemize}